\documentclass[preprint, 3p,
authoryear]{elsarticle} %review=doublespace preprint=single 5p=2 column
%%% Begin My package additions %%%%%%%%%%%%%%%%%%%

\usepackage[hyphens]{url}


\usepackage{lineno} % add

\usepackage{graphicx}
%%%%%%%%%%%%%%%% end my additions to header

\usepackage[T1]{fontenc}
\usepackage{lmodern}
\usepackage{amssymb,amsmath}
\usepackage{ifxetex,ifluatex}
\usepackage{fixltx2e} % provides \textsubscript
% use upquote if available, for straight quotes in verbatim environments
\IfFileExists{upquote.sty}{\usepackage{upquote}}{}
\ifnum 0\ifxetex 1\fi\ifluatex 1\fi=0 % if pdftex
  \usepackage[utf8]{inputenc}
\else % if luatex or xelatex
  \usepackage{fontspec}
  \ifxetex
    \usepackage{xltxtra,xunicode}
  \fi
  \defaultfontfeatures{Mapping=tex-text,Scale=MatchLowercase}
  \newcommand{\euro}{€}
\fi
% use microtype if available
\IfFileExists{microtype.sty}{\usepackage{microtype}}{}
\usepackage[]{natbib}
\bibliographystyle{plainnat}

\ifxetex
  \usepackage[setpagesize=false, % page size defined by xetex
              unicode=false, % unicode breaks when used with xetex
              xetex]{hyperref}
\else
  \usepackage[unicode=true]{hyperref}
\fi
\hypersetup{breaklinks=true,
            bookmarks=true,
            pdfauthor={},
            pdftitle={La domination du monde par l'intelligence artificielle, ça commence par la science},
            colorlinks=false,
            urlcolor=blue,
            linkcolor=magenta,
            pdfborder={0 0 0}}

\setcounter{secnumdepth}{5}
% Pandoc toggle for numbering sections (defaults to be off)


% tightlist command for lists without linebreak
\providecommand{\tightlist}{%
  \setlength{\itemsep}{0pt}\setlength{\parskip}{0pt}}






\begin{document}


\begin{frontmatter}

  \title{La domination du monde par l'intelligence artificielle, ça
commence par la science}
    \author[Université de Sherbrooke]{Rosalie Gagnon%
  %
  }
   \ead{rosalie.gagnon@usherbrooke.ca} 
      \cortext[cor1]{Corresponding author}
  
  \begin{abstract}
  Dans un monde ou la technologie avance à une vitesse lumière, un
  nouvel arrivant nous confronte. ChatGPT la nouvelle intelligence
  artificielle, fait des vagues. Versatile, il est garanti qu'il peut
  trouver une réponse à toutes vos questions les plus bizarres. Le monde
  de la science appréhende ce nouvel arrivant porteur d'un vent de fakes
  news. Utilisé à des fin malicieuses, ChatGPT pourrait être un désastre
  pour la communauté scientifique. Cependant, qui dit mauvais peut aussi
  dire bon. ChatGPT condamnera-t-il la science ou la fera-t-elle
  progresser comme jamais auparavant?
  \end{abstract}
    \begin{keyword}
    ChatGPT \sep 
    Science
  \end{keyword}
  
 \end{frontmatter}

\hypertarget{introduction}{%
\section{\texorpdfstring{\textbf{Introduction}}{Introduction}}\label{introduction}}

La science-fiction a, depuis bien longtemps, prédit l'avènement des
intelligences artificielles. Qu'on prévoit la destruction du monde par
celle-ci ou qu'on questionne la nature humaine même, l'intelligence
artificielle est un miroir déformé de son créateur, l'humain. Si dans le
célèbre roman `Les androïdes rêvent-ils de moutons électriques' publié
en 1968 par l'auteur Philip K. Dick, on ressasse une réalité
futuristique ou l'on se questionne sur l'essence même de ce qui fait de
nous des humain, la réalité du siècle est certainement différente.
Aucune intelligente artificielle n'est encore assez développée pour
qu'on se pose de tels questions. Cependant, avec l'apparition du nouveau
chatbox appelé chatGPT des questions font surface. Qu'est-ce que cette
intelligence peut causer comme impact sur la communauté scientifique? Un
peu moins dramatiques que les questions existentielles proposées par
`blade runner', ce sont tout de même des questions qui nécessitent qu'on
s'y penche. Si on lui demande, ChatGPT peut composer n'importe quel
texte qui semble être écrit de la main humaine. Gagnant en popularité
comme aucun autre site auparavant, il soulève avec lui, une vague de
scientifique inquiets face aux impacts que peut provoquer cette nouvelle
interface sur le monde de la science.

\hypertarget{chatgpt-un-a.i-versatile}{%
\section{\texorpdfstring{\textbf{ChatGPT, un A.I
versatile}}{ChatGPT, un A.I versatile}}\label{chatgpt-un-a.i-versatile}}

Mais qui est ce mystérieux A.I qui menace le monde de la science?
ChatGPT est une intelligence artificielle présentée sous forme de
chatbox. On peut lui demander n'importe quoi et il répondra de façon
presque indiscernable à celle d'un humain. Que ce soit un poème, une
liste d'épicerie, le rendement des cultures de soya en 2019 ou à quel
moment de la journée tu devrais prendre tes vitamines, ChatGPT fournira
une réponse. Hormis quelques garde-fous éthiques très vagues, Chat
fourni presque en tout temps une réponse crédible et confiante. C'est là
que réside une partie du problème. Selon un article publié dans le
`American journal of bioethics' le programme offre des réponses
instantanées à des questions complexes sans autres alternatives à
visiter au contraire d'un moteur de recherche comme google par exemple,
qui lui fourni plusieurs liens à consulter.\citep{doshi_chatgpt_2023}
Plusieurs utilisateurs du programme pourraient être amenés à croire tout
ce qu'il leur est donné en réponse par celui-ci. De plus, son fin talent
à reproduire ce qui semble avoir été écrit de toute pièces par un homme
ou encore de réécrire un texte dans un différent style rend la détection
du plagiat presque impossible. On craint ainsi qu'un flot de d'articles
scientifiques générés par le chatbox se déverse sur le net, dupant la
population avec sa crédibilité de fer \citep{noauthor_ai_nodate}. Sans
être obligé de générer un article au complet, certains utilisateurs
peuvent également utiliser ChatGPT pour un renseignement ou de l'aide
dans leurs recherches. Ici, le danger est d'obtenir une réponse qu'on
interprète mal ou qu'on ne comprend tout simplement pas et de l'utiliser
dans un papier ou comme référence. Si on couple ce phénomène avec les
sites de publication scientifique open access qui ne nécessitent pas de
revues par les pairs avant la publication, nous avons alors à faire à un
enfer de fausses informations à venir.

\hypertarget{contrer-les-pouvoirs-infinis-de-chatgpt-par-laugmentation-de-la-reproductibilituxe9-des-uxe9tudes}{%
\section{\texorpdfstring{\textbf{Contrer les pouvoirs infinis de chatGPT
par l'augmentation de la reproductibilité des
études?}}{Contrer les pouvoirs infinis de chatGPT par l'augmentation de la reproductibilité des études?}}\label{contrer-les-pouvoirs-infinis-de-chatgpt-par-laugmentation-de-la-reproductibilituxe9-des-uxe9tudes}}

Dans la communauté scientifique, on reconnait qu'il est presque
impossible de reproduire exactement l'étude d'un autre scientifique en
obtenant plus ou moins les mêmes résultats dans un intervalle de
confiance raisonnable. Même les scientifiques qui tentent de reproduire
leurs propres études peinent à arriver aux mêmes conclusions que
précédemment. Selon un article de de Monya Baker publié en 2016, près de
60\% des scientifiques en biologie on échoués à reproduire leurs propres
études et 78\% on échoués à reproduire celles d'un autre chercheur
\citep{open_science_collaboration_estimating_2015}. Ce phénomène est dû
en partie aux méthodes de travail utilisés. Pour avoir une étude
reproductible, il est important de prendre en note chaque changement
dans le code, chaque faux pas dans les manipulations, chaque
modification de contexte d'interprétations des données dans le but
ultime qu'un chemin clair et précis soit pointé. Ainsi ce chemin est
facilement empruntable par d'autre scientifiques sans risque d'égarement
ou de divergence ce qui mène optimalement à la même destination
c'est-à-dire, les mêmes résultats et conclusions. C'est un détour que
peu de scientifiques se donnent la peine de prendre puisqu'il demande en
général plus de temps et d'argent que lorsqu'on utilise la méthode
standard de noter seulement les grandes lignes d'une méthodologie. De
plus, peu de scientifiques sont enclins à reproduire des études déjà
faites puisque les papiers découlant de ces études sont infiniment moins
enclins à se faire publier dans les revues ou les journaux. Cependant il
serait avantageux pour la communauté scientifique d'adopter une éthique
de travail qui facilite la reproductibilité puisque non seulement un bon
pourcentage d'études publiés comprennent des erreurs et quelques fois de
la falsification de données mais la reproductibilité pourrait servir
comme un outil efficace contre l'engouement vers ChatGPT. Si notre
chatbox national invente ses données et ses statistiques de toutes
pièces basées sur n'importe quelles stupidités prises sur le net, la
reproductibilité des études issues de celui-ce sera de zéro. Il sera
ainsi plus facile de discerner un article honnête d'un article forgé par
le bot. Il est donc avantageux pour tout le monde d'augmenter son taux
de reproductibilité puisque les études en deviennent plus crédibles et
on peut discerner les vraies des fausses.

\hypertarget{dompter-la-buxeate-et-sen-servir-uxe0-bon-escient}{%
\section{\texorpdfstring{\textbf{Dompter la bête et s'en servir à bon
escient}}{Dompter la bête et s'en servir à bon escient}}\label{dompter-la-buxeate-et-sen-servir-uxe0-bon-escient}}

Mais, si chatGPt est si puissant, serait-il possible d'envisager une
avenue où il pourrait être notre allié scientifique? Eh bien déjà
plusieurs scientifiques s'en servent pour faire des revues d'articles,
les aider à reformuler des paragraphes, ou autres petites tâches
simplettes. Dans un article du journal Nature, une scientifique dit voir
un avenir ou ChatGPT aide les scientifiques à devenir plus productifs au
niveau de la recherche. ChatGPT a aussi été recconu pour son aide dans
le domaine de la santé. Puisque le bot peut se concentrer sur l'analyse
des données, les revues de littérature et autres tâches plus basiques,
il laisse le temps aux chercheurs de se pencher sur la découverte et la
compréhension de nouveaux médicaments \citep{sallam_chatgpt_2023} ce qui
est une aide considérable. Mais allons voir plus loin pour un instant,
serait-il possible qu'un jour, avec une stadardisation de la
reproductibilité en science assez poussée et une éthique de travail
impeccable ChatGPT puisse se pencher sur le travail faramineux de
reproduire les études déjà produites? Ainsi, plus besoin de `perdre son
temps' à reproduire les études des autres. Les chercheurs peuvent se
pencher sans cesses sur de nouveaux sujets pendant que la crédibilité de
leurs papier publiés dans le passé sont vérifié par ChatGPT.

\hypertarget{conclusion}{%
\section{\texorpdfstring{\textbf{Conclusion}}{Conclusion}}\label{conclusion}}

ChatGPT n'est sûrement pas la dernière intelligence artificielle que la
science devra confronter. Autant des enjeux négatifs font surface comme
le risque de plagiat, de falsification ou de publications de faux
articles en masse sur le net, il y a aussi des avantages bénéfiques tel
que l'aide à la recherche, à la synthèse de texte, à la revue de
littérature ou à l'interprétation des données qui ont été mise de
l'avant par ce nouveau programme. Finalement, il serait important de
décréter une éthique de travail face à l'utilisation de ChatGPT en
recherche scientifique avant de l'utiliser allègrement dans le domaine.
Pour l'instant, il est important de faire preuve de transparence et
d'augmenter son taux de reproductibilité le plus possible dans chaque
recherche en attendant qu'une direction d'utilisation soit pointée. Qui
sait, peut-être que, mis de paire avec une reproductibilité infaillible,
chatGPT pourrait faire avancer la science comme jamais il à été vu
auparavant, mais attendant nous devons être prudent avec ce nouvel outil
à double tranchant. J'ai donc une opinion ni pour ni contre mais qui se
veut vigilante en premier lieu mais ultimement je crois que Chat peut
devenir un outil d'une grande utilité en science.

\renewcommand\refname{References}
\bibliography{mybibfile.bib}


\end{document}
